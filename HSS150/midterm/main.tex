\documentclass{homework}
% \usepackage{lua-visual-debug}
\usepackage{amsmath, amssymb, amsfonts}
\usepackage{ulem}
\usepackage{macros-common}
\usepackage[bb]{macros-prob}

\title{Midterm}
\subject{HSS150 Introduction to Politics}
\studentid{20170058}
\name{Keonwoo Kim}
\date{\today}

\setstretch{1.04}

\begin{document}
\maketitle

\solution{\textit{Classical liberalism perceives state as a “necessary evil.” However, modern liberals are sympathetic toward state intervention. Based on the concept of negative and positive liberty, explain the difference between classical and modern liberalism with respect to the role of the state. Do you agree with the claims of modern liberals? Pick yes or no and explain why.}}

Classical liberals see individuals in the atomist view: it means the society is based on the negative liberty, which means the state and the government should provide a minimal level of intervention and there should not be any kinds of external constraints on each person. Though in a classical liberal's viewpoint, the state is yet necessary in meaning that it has a responsibility to found a social order, protect people, and make pledges and agreements work properly. However, it is evil because every law and regulation of the state act as a press on individuals, and it should be avoided to guarantee the maximal level of individual liberty. Classical liberalism shares a common philosophical approach as economic liberalism, which insists on laissez-faire capitalism, letting the free market do everything without any interference from the government.

On the other hand, modern liberalistic society is based on the positive liberty: leaving the society alone to work does not guarantee the individual liberty, rather the state should intervene the society to assure the ability of individuals to improve and flourish themselves so that they can have the "liberty" of the true meaning. And this leads to the state offering social welfare, as Keynesian economics says the capitalistic economy of a state could last long only if it is managed by some regulations like welfare and redistribution.

I agree with modern liberals in that individual liberty can be materialized only if some sort of assertion to live well is backed by social welfare. Extreme poverty, an absence of shelter to live in, or having no medical insurance limit greatly one's survival and hence one's liberty. And we had experienced concrete examples of the end of laissez-faire capitalism resulting in an extreme inequality harming many people's liberty. Therefore, this issue of society should be fixed by regulating the market appropriately and not making the government too small to accomplish that regulation. 

\newpage

\solution[3]{\textit{Neo-Marxists argue that capitalist democracy is inherently unstable and the “legitimation crisis” is unavoidable. What is the tension between capitalism and democracy? Why is the legitimation crisis unavoidable? Do you think the concept of “legitimation crisis” useful to explain the political economy of modern industrial society? }} 

The basic idea of democracy is in political equality, meaning that political power should be distributed in an equal way. For instance, anyone should have equal rights to vote or act politically. However, in a capitalistic and meritocratic society, there must be some kinds of inequality in wealth and authority. Thus, these make the tension between democracy and capitalism.

Furthermore, with liberal democratic ideology, the state should provide welfare to state members. The better life quality to ensure for working or non-bourgeois classes, the more welfare should be guaranteed. As the level of welfare increases, however, it imposes a burden on the budget of the government. This leads to giving high tax burden to enterprises, and consequently, this reduces the competitiveness of companies so that it negatively affects the capitalistic economy. This is another tension between capitalism and modern democracy based on liberalism. Hence, capitalism and liberal democracy are contradicting each other. Because of these tensions, neo-Marxists insist that capitalist democracy cannot last long and the legitimation crisis of the capitalist democratic society becomes ineradicable.

Though it might not enough to predict the future, the concept of legitimation crisis can be useful to observe tensions between democracy and capitalism presented in our society. The tensions discussed above help us to face up to the main issue we need to care about while living in a capitalistic democratic society. Besides, this emphasizes that it is important to keep a certain level of welfare in such a society to prevent it from collapsing. 


\newpage

\solution[4]{\textit{Explain Ernest Gellner’s claim that nationalism is a product of modernization. Construct an argument against Gellner’s statement based on Hobsbawm’s claim that nationalism is an “invented tradition.” Do you agree with Hobsbawm’s explanation of nationalism?}}

Gellner claims that industrialized society made nationalism naturally. In western countries, before modernization and industrialization, the society was feudalistic and many people were farmers. So people's jobs were associated with their living area, and people were tied into one with a network of such bonds. However, the more developing industrial society, the more people were started to move and hence the level of social mobility increased. And society urges people into competitions. This leads to the introduction of new social cohesion, nationalism, in the sense of ethnicity, language, or the land on which people are standing. This introduction is unavoidable in Gellner's theory.

However, Hobsbawm tried to see the nationalism in the other way round: a faith in long continuous history and cultural purity of a nation is created by the nationalism. In this point of view, Gellner's claim that the rise of industrialization introduced nationalism is wrong as there was no continuous history and pure culture that was shared with members of the society before they were invented.

I agree with Hobsbawm's explanation. There are many ``nations'' that promotes nationalism to fulfill their political desire. Some of those even modify the history to rationalize nationalism. Though it may not be linked directly with economical reasons, there is a somewhat political issue on any nationalistic topic. For instance, nationalism could be related to the independence of a nation if it is under a colonial country. As another example, some racist people are insisting on their white purity to prevent immigrants from coming into the US. These kinds of nationalism are unnatural with industrialization and modernization, and can only be accounted for political reasons which are made up to consolidate nationalism. 






\end{document}