\documentclass{homework}
% \usepackage{lua-visual-debug}
\usepackage{kotex}
\usepackage{amsmath, amssymb, amsfonts}
\usepackage{ulem}
\usepackage{hyperref}

\setmainhangulfont{NotoSerifCJKKR}

\title{\LARGE Can we bring both capitalism\\[.3em] and liberal democracy?}
\subject{HSS150 Introduction to Politics}
\studentid{20170058}
\name{Keonwoo Kim}
\date{\today}

\setstretch{1.2}

\begin{document}

\maketitle


Throughout the past two centuries, capitalism dominated the most of the world, following the demise of Soviet socialism and the transition of (People's Republic of) China's economy into capitalistic one. One can say China has been a capitalistic nation while the Chinese constitution states that The PRC is a socialist state. Only few conturies such as North Korea resists to accept the capitalism and keeps the socialistic regime using brutal power of government. Considering this enormous success of capitalism, the vast majority of the world will take capitalism for a long time, or maybe forever.

Also, democracy became the most major political system in the world, especially from the late twentieth century. After the Cold War, many centralized authoritarian states in Soviet Union disappeared and have changed into democratic nations. These discludes some nations like China and North Korea, while they both insist that they are led by people's democratic dictatorship. When we take the minimal standards of democracy as a measurement, there were 123 conturies, out of around 200, which is sort of taking an electoral democracy, as of 2010 \cite{FreedomHouse2010}.

South Korea also accepts both spirits of capitalism and democracy, especially the liberal one, after some coups d'\'etat and people's movements. At this moment, capitalism and liberal democracy settled in South Korea as major systems of the nation. Recently, there are many dicussions of political issues such as residential policy and Corporate Homicide Act-like laws. Na\"ively, on the capitalistic side, one will pursue economic interests and hence object to a huge tax burden and the establishment of such laws regulating companies. And on the other side---liberal democratic side, one will value the equal protection of human rights, civil rights and civil and political freedoms, including residential rights. So they are seemingly conflicting with each other.

This kind of discussion extends to what is called the legitimation crises, introduced by a German socialogist J\"urgen Habermas in his book \textit{Legitimation Crisis}. According to him, this means that a capitalistic democratic state is difficult to maintain the legitimacy, which is one of crucial properties of a state, because a capitalistic democratic state is forced to resist the democratic pressure or experience an collapse of the capitalist economy. In such a society, working class will make a pressure to increase welfare spending of the state so that the quality of the life of working class can be improved. This will lead to higher tax revenue and hence it contradicts to the idea of capitalism. Thus, the state will lose its legitimacy, following both ideology of the state.

Moreover, it is difficult to change the system of a state. One can say that the system is just a collection of people, so that when people want to change the system, the turnover can be easily achieved. However, by the theory of the paths of least resistance in sociology, systems (of privilege) are not just collections of people. Note that it is undeniable that individuals make social systems work. What is more in social systems is that, as people participate in systems, their lives are shaped by socialization and paths of least resistance. 

Paths of least resistance mean the way people behave in a social system minimizing the conflict between the system and themselves. For instance, there are some people laughing at raist or sexist jokes even when it makes them feel uncomfortable, especially taking with people at a \textit{higher} position. In that situation, to not laugh and rist being ostracized by everyone may make them more embarassed. The most convenient way is to go along.

As an another example, in the game Monopoly, we notice certain patterns of feeling and behavior that reflect paths of least resistance inherent in the game itself. The game encourages to feel good when other players land on my properties and be deprived of their assets, and to feel anxiety about landing on others' property. In this game, such behaviors are the examples of paths of least resistance. Paths of least resistance usually reinforce the system to survive and control individuals again. 

This is why changing a system is difficult than expected. Clearly, it can be applied to both capitalism and (liberal) democracy. Therefore, unless there are extremely severe problems in those systems, such as a legitimation crisis, it seems a current-liberal-capitalistic state will protect capitalistic democratic value for the entire remainder of its existence.

Then, the question now becomes clear: can capitalism and liberal democracy be compatible with each other without any severe problems? I think it is possible, as seen merely from many nations until now. It is because of the following two reasons.

First, there exist capitalist policies which reinforce liberal democracy. Actually, the central objective of contemporary public policies is to ensure that contemporary capitalism strengthens liberal democracy. The following are examples of these policies:

\begin{itemize}
  \item Nowadays, it is almost essential to receive secondary education or some post-secondary training in order to do a job which robots and machines are not able to do. Therefore, to develop the economy for the future, one should provide (post-)secondary education to the public. This agrees with the concept of educational right of the democratic ideology.
  \item Entities in capitalism can be developed by competitions with others. However, in most advanced societies, economic concentration and regional inequality are increasing. These obviously discourage them from being competitive and innovative and hence it deprives a chance to develop capitalistic society. Thus, to advance a capitalistic society, one would need to solve the disparity and concentration problems, which agree with democratic values.
\end{itemize}

Second, in contemporary society, capitalistic values are changing to suit into democratic values. It is related to what is called ethical consumerism. Ethical consumerism is a type of consumer activism to buy ethically-made products. In many cases, those ethical values agree with democratic values.

In the sense of equality, we may think of fair trade. Fair trade is an activism that urges to pay a reasonable and fair price for the products in global trade, especially for goods that is exported from developing conturies to developed conturies. Examples of such products include coffee beans, cacao beans, or cottons. Basic principles of fair trade are as follows: making chances to work for economically disadvantaged producers, better working conditions, gender equity, keeping children's rights---such as well-being, security, educational requirements and need for play---regarding child labor, and so on.

As another example, we can consider the case of Namyang dairy products company. Namyang is infamous for its dishonesty due to a bunch of wrongdoing. To name some of them, they forced retail stores to buy their products and cancelled a contract with those retail stores if they rejected to do so, and they fired pregnant female employees while they sell powdered formula for babies. Those consecutive events make Namyang lose their credibility and it affected directly to the sales. \cite{Namyang}

As an extension to the case above, people started to consume ethically based on gender equality and feminism. In case of Kolmar Korea, a cosmetic and pharmaceutical company, the CEO of the company forced employee to watch a video devaluing women and government's foreign policy to Japan. After this, the CEO resigned and the stock price had havled in 2019. \cite{Kolmar} Other companies such as Hanssem, Seoul milk, several banks and news media outlets are also criticized from this perspective. As some words like \textit{she-conomy} or \textit{fem-vertizing} coined, capitalism concerns (and should concern) the gender sensitivity.

Another important topic in South Korea is the climate crisis. Consecutive natural disasters in this year and a rapid increase of plastic trashes because of the appearance of COVID-19 sparked the discussion about the climate change and global warming. And people started to have interests about eco-friendly, plastic-free, and sustainable products. As the climate crisis affects critically to the well-being of people and the sustainability of people's life, it is closely related to the democratic value. And the capitalism responds by producing their products sustainably and without much waste, or making the packages for their products have less plastic components. \cite{Eco,Eco2}

Moreover, last year, European Union introduced Carbon Border Tax to protect eco-friendly companies from other companies which have cheap and competitive prices due to weak environmental regulation. This introduction of tax burden on environmental destruction is based on \textit{eco-capitalism}, which is the view that capital exists in nature as ``natural capital'' on which all wealth depends.

Veganism is also a big part of ethical consumption. As importances of climate change, health issue regarding modern dietary lifestyle, animal rights, and animal welfare rise, veganism have become one of ethical lifestyle. Some convenience stores in South Korea presented vegan packed lunches, and E-mart introduced the vegan zone in some shops. Some cosmetic brands have shown \textit{cruelty-free} products to respond to it.

As the power of consumer group grows, capitalism recognizes and accepts democratic values into capitalistic values by either government policies or people's consumption, as seen from the cases described above. This means one can pursue capitalistic and democratic values simultaneously, and so does a state.

Seeing from these empirical evidences, capitalism and liberal democracy may coexist in contemporary society. It is good to watch carefully about the tensions between capitalism and liberal democracy as Habermas argues that there can be legitimation crisis, the state will not collapse when we develop both ideologies in a non-conflicting way.



\begin{thebibliography}{9}
  \bibitem{FreedomHouse2010} 
  Freedom House. \textit{Freedom in the world.} 2010. \url{http://www.freedomhouse.org/report/freedom-world/free-dom-world-2010#.Uu_OLrS2yF8}
  \bibitem{Namyang} ChosunBiz. \textit{'1등 분유'에서 '불매기업'으로 추락\dots남양 잔혹사}. 2019. \url{https://biz.chosun.com/site/data/html_dir/2020/05/09/2020050900083.html}
  \bibitem{Kolmar} 한겨레. \textit{무섭게 타오른 불매운동, 한국콜마 회장 결국 사퇴했지만\dots}. 2019. \url{https://www.hani.co.kr/arti/economy/consumer/905338.html}
  \bibitem{Eco} 조선일보. 친환경 아니면 퇴출되는 시대, 기업들 인증 바람. 2020. \url{https://www.chosun.com/site/data/html_dir/2020/08/25/2020082500322.html}
  \bibitem{Eco2} 식품외식경제. [2020 신년특집⑩] 식품·유통업계, 친환경 포장은 필수. \url{https://www.foodbank.co.kr/news/articleView.html?idxno=59176}
\end{thebibliography}

\end{document}