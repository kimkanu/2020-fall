\documentclass{homework}
% \usepackage{lua-visual-debug}
\usepackage{amsmath, amssymb, amsfonts}
\usepackage{ulem}
\usepackage{macros-common}
\usepackage[bb]{macros-prob}

\title{}
\subject{HSS150 Introduction to Politics}
\studentid{20170058}
\name{Keonwoo Kim}
\date{\today}

\setstretch{1.04}

\begin{document}

\vspace*{1cm}

\begin{center}
  \Large \bf Tensions between Liberal Democracy and Capitalism in South Korea
\end{center}
\vspace*{1cm}

My term paper will deal with tensions between liberal democracy and capitalism and the concept of the legitimation crisis. Through this paper, I will investigate some examples of tensions between liberal democracy and capitalism surfacing nowadays. Based on that, I'll try to argue that it is merely impossible for individuals to get out of capitalistic suppression pursuing capitalism, and hence the legitimation crisis occurs hardly unless under some extreme situations such as an economic collapse, as opposed to the claim of neo-Marxist, at least in South Korea.

I will impose the theory of paths of least resistance in sociology to insist that it is quite difficult to make a change in social conventions. This will be backed with the examples of the events happening until now in South Korea and other countries.
Further, the complete liberation of individuals from capitalistic value contradicting the value of liberty is not going to happen in capitalistic liberal democratic society. In my opinion, historically, even if there are some changes in social conventions, those conventions and also those changes are backed by a certain capitalistic value. For instance, recent attentions on environmental issues and climate crisis are due to natural disasters being more frequent as years go by, which bring more losses to agricultural and residential stuffs. Therefore, it is truely difficult to be free from capitalism.
Finally, those capitalistic values continuously bother individuals from get a full liberty in the sense of liberal democracy, while people tends to bear them following the law of paths of least resistance.

This topic is closely related to the course material, where democracy, capitalism, the legitimation crisis were somewhat central concepts dealt in the class.




\end{document}