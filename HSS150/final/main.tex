\documentclass{homework}
% \usepackage{lua-visual-debug}
\usepackage{amsmath, amssymb, amsfonts}
\usepackage{ulem}
\usepackage{macros-common}
\usepackage[bb]{macros-prob}

\title{Final Exam}
\subject{HSS150 Introduction to Politics}
\studentid{20170058}
\name{Keonwoo Kim}
\date{\today}

\setstretch{1.04}

\begin{document}
\maketitle

\solution{\textit{Describe the tension between pluralist approach to multiculturalism and liberal individualism. Evaluate the strength and the weakness of pluralist version of multiculturalism.}}


The pluralist approach to multiculturalism is based on the idea of value pluralism, which insists that there is no single, overriding conception of the "good value." According to it, there are many kinds of values coexisting which are competing and equally legitimate. For instance, democracy and personal freedom are just some examples of competing values and they have no greater moral authority than other beliefs. Regarding this, pluralist multiculturalism can point out that liberal point of view to multiculturalism is biased into putting emphasis on individual liberty, which is quite a western belief. Thus, this pluralist approach can provide a criticism to the liberal way of thinking regarding multiculturalism, and it focuses on minority cultures which are oppressed by liberal and western ideology. However, pluralist multiculturalism does not give an answer to the way several contradicting cultures coexist, such as how a highly liberal culture and an illiberal culture can coexist. This is a crucial weakness of pluralist argument to multiculturalism.

\newpage

\solution[3]{\textit{Korean political system frequently fails to mediate conflicting major social interests. Progressives in Korea believe that we need to learn from Western European model of corporatism. Do you agree with them? Pick yes or no and justify your position.}} 

I personally disagree on the view that we need to accept Western European model of (neo-)corporatism as it is. Corporatism has been criticized due to problems in its nature. First, as in the form of tripartitism, corporatism does not care about customer groups or promotional groups. And also the previlige to be included in the discussion with government and business groups is restricted to few representative organizations and groups. Second, under corporatism, interest groups are often hierarchically ordered and governed by some leaders, but those leaders are usually not responsible for each member. Thus, corporatism may act as a system to control the society. Moreover, usually leaders of corporatism are not elected through a representative process, and this might harm representative democracy of contemporary society. And corporatism is too weak to keep the neutrality and equality of interest groups; for instance, government can be captured by a specific interest group, and in this case it is difficult to make a politically right decision. Finally, one of the representative Western European models of corporatism, the Sweden model, has been criticized due to its low economic growth and its uncertain subsistence. As more and more globalized the society becomes and more and more divided the labor becomes, the main fault of those Western European model is being firm. It is why I think slight modifications of (neo-)corporatism will lead to such a problem eventually due to its nature. Therefore, it would be better not to accept corporatism as a resolution of conflicts in major social interests.


\newpage

\solution[4]{\textit{According to the liberal theory of law, judges are supposed to be independent from politics. However, many people believe that the some court decisions are heavily affected by political factors. Levitsky and Ziblatt point out that some political leaders try to capture the judicial branch in favor of the ruling party. Do you think the court is politically biased? Pick either yes or no and explain why.}}

There are certain biases in many aspects of the court, at least in the cases of the United States and South Korea. Clearly, the court is political in the sense that it affects the political system of the society.
As an external bias, we can think of the recruit of supreme court judges, which is done by president and confirmed by the legislature. Therefore, especially if the ruling party is taking a major position in the legislature, it would likely have some kinds of biases. For a concrete example, after Ginsburg's death, D. Trump is trying to nominate conservative politician in the supreme court, while the trial of Obama government 4 years ago was blocked. As seen from that Trump tried to decide the voter fraud issue in the US supreme court, it seems obvious that he actually wanted the political bias of the court and tried to reinforce it by nominating his ally. Therefore, the nomination of supreme court judges are politically biased. Since there are quite few judges in the supreme court, the change of one judge can be critical to its political neutrality.
As an internal bias, we may think of socialists' and feminists' arguments. Since judges are usually in the elite groups, they might be biased toward conservatism, according to socialists' opinions. On the other hand, there are gender inequality in many places of the society, the judiciary as well. Women's participation in judicial system is, on average, almost a half of men's. In case of South Korea, the problem is more severe that the number of female professional judges less than 30% of the number of male judges, according to an OECD survey. This can cause another internal bias in the judiciary system.
To sum up, there are political biases in the court as some empirical evidences of them were observed.




\end{document}