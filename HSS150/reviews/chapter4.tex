\documentclass{homework}
% \usepackage{lua-visual-debug}
\usepackage{amsmath, amssymb, amsfonts}
\usepackage{ulem}
\usepackage{macros-common}
\usepackage[bb]{macros-prob}

\title{Review: Chapter 4}
\subject{HSS150 Introduction to Politics}
\studentid{20170058}
\name{Keonwoo Kim}
\date{\today}

\setstretch{1.04}

\begin{document}

\maketitle

The authors are trying to investigate how do the authoritarians, even elected politicians via demotratic processes, subvert democracy--in particular, how do they destruct democratic institutions opposing them in order to elongate their one-man rule.

First, captures ``neutral" institutions, which should capture governments' wrongdoing. Usually, it is done by firing nonpartisan people, or increasing the size of the agency (court packing), and filling with loyalists. In Hungary, Viktor Orban made judicial systems full of partisan allies. It can be done by giving bribes or posing a threat. Montesinos, Fujimori's advisor, did this well. The most extreme way is to discard the courts altogether and make a new one, as Chavez did.

After making referees into allies, they try to sideline opposite politicians, business leaders, major media, and cultural figures. One way is to buy those opponents off. Fujimori government and Montesinos bought every media and some opposite politicians, such as Luis Chu, Miguel Ciccia, and Susy Diaz.

If someone is not able to be bought, they can be jailed, exiled, or filed a suit. Under Peron, opposition leader was imprisoned for disrespecting president. Similar thing happened in Malaysia and Venezuela. Opposition media can be suited, as El Universo, Dogan Yayin media conglomerate, and NTV television did. When key media are assaulted, others censor themselves, as Venevision did.

Autocrats also try to weaken business leaders opposing the government, as Putin summoned wealthiest businessmen and warn them. Some of them, including Berezovsky and Khodorkovsky were ordered to be arrested by Putin.

Further, they try to quieten cultural figures getting great popularity. Usually, they choose to make cultural figures far from politics. As Dudamel criticizes the Chavez government, the government cancelled planned Orchestra to the US.

Thirdly, rewrite the system for them to be privileged, as in the Malaysian election system is gerrymandered to make it advantaged to the ruling party. In the US, the voting law is changed to be discriminatory to black people.

Finally, they use international crises to gain popularity. For instance, aftermath of 9/11, Bush was his approval rating soar from 53 percent to 90 percent, and similarly with Roosevelt's internment of Japanese-Americans.

The article had many examples so that the argument seemed logical. And the transition of those processes are smooth, so it was a well-formed article.

However, some of arguments were not universally applicable. For instance,  the reason why division into old and contemporary was made was unclear.

\end{document}