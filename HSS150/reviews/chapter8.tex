\documentclass{homework}
% \usepackage{lua-visual-debug}
\usepackage{amsmath, amssymb, amsfonts}
\usepackage{ulem}
\usepackage{macros-common}
\usepackage[bb]{macros-prob}

\title{Review: Chapter 8}
\subject{HSS150 Introduction to Politics}
\studentid{20170058}
\name{Keonwoo Kim}
\date{\today}

\setstretch{1.04}

\begin{document}

\maketitle


In this chapter, the authors are dealing with Donald Trump's effort to harm democracy, and the fate of democracy during the rest presidency of him.

Trump attempted all three strategies of authoritarians to break democracy described in Chapter 4. First, he tried to capture the referees. He attempted to intervene FBI's investigation into Russia ties of Trump's campaign, and pressed the heads of intelligence agencies. He also punished agencies behaving independently. Besides, he attacked other agencies which are not loyal such as judicial agencies and OGE.

To sideline key players, he used rhetorical attacks to media outlets such as NY Times, CNN, and NBC. As Rafael Correa did, he also threatened media companies with libel suits and government regulation.

To tilt the ground, one thing he did is to suppress Democrats voters. He established PACEI, which insisted voter fraud and passed strict voter ID laws in some states.

Whether the eventual fate of Trump government is dependent on the behaviour of leaders of ruling party, public opinion, and crisis. For Republican leaders, they could choose one of the following: remaining loyal, containment, or seeking Trump's impeachment. The authors said Republicans responded with a mix of loyalty and containment.

Public opinion is also important because authoritarians not having strong violences should gain public support. It will depend on the economy of the nation, as well as accidental events, such as crises.

Security crises can turn over the political situation, and people usually increase popularity for the government, becoming more patient or supportive. Those crises also give executive power to government, which can harm democracy.

Finally, the authors said Trump has been breaking societal norm in a bad way. In some cases, norm breaking could be not harmful. However, Trump has broken unwritten rules in totally different sense: appointed relatives, maintained his business, yelled obvious untruths, abandoned political civility, insulted media outlets publicly, and so on. The authors insisted those rule-breaking will bring terrible consequences.

The discussion has strength in its consistency with one in Chapter 4 and lots of empirical evidences done by Trump. It provides a persuasive reasoning on why Trump is dangerous to democracy.

One thing went astray is that COVID-19 pandemic had decreased Trump's popularity from April to July due to his attitude to it. And, The discussion spotlights the dangerousness of Trump, but not its persistence. As many authoritarians collapsed, it could be applied to Trump, too.

\end{document}