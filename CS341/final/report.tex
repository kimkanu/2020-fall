\documentclass{homework}
% \usepackage{lua-visual-debug}
\usepackage{amsmath, amsfonts, amssymb}
\usepackage{enumitem}
\usepackage{graphicx}
% \usepackage[a4paper, total={6in, 8in}]{geometry}

\title{Final Exam}
\subject{CS341 Introduction to Computer Networks}
\studentid{20170058}
\name{Keonwoo Kim}
\date{\today}

\setstretch{1.0}

\begin{document}
\maketitle

\parindent=0pt

\solution{
  {[1]} T\\
  {[2]} T\\
  {[3]} F\\
  {[4]} T\\
  {[5]} F\\
  {[6]} F\\
  {[7]} F\\
  {[8]} F\\
  {[9]} F\\
  {[10]} T\\
}

\solution{
  \begin{enumerate}[label={\arabic*)}]
    \item Interface 0: \texttt{00\_\_\_\_\_\_}, i.e., from \texttt{00000000} to \texttt{00111111};\\
    Interface 1: \texttt{010\_\_\_\_\_}, i.e., from \texttt{01000000} to \texttt{01011111};\\
    Interface 2: \texttt{011\_\_\_\_\_} and \texttt{10\_\_\_\_\_\_}, i.e., from \texttt{01100000} to \texttt{10111111};\\
    Interface 3: \texttt{11\_\_\_\_\_\_}, i.e., from \texttt{11000000} to \texttt{11111111}.

    \item a. The devices at the rightmost position in the figure have addresses \texttt{192.168.1.1}, \texttt{192.168.1.2}, and \texttt{192.168.1.3}. And the gateway of the router to the home network is \texttt{192.168.1.4}.
    
    b. 
    \begin{center}
      \begin{tabular}{cc}
      WAN side & LAN side \\ \hline
      \texttt{24.34.112.235, 5001} & \texttt{192.168.1.1, 3345}\\ 
      \texttt{24.34.112.235, 5002} & \texttt{192.168.1.1, 3346}\\ 
      \texttt{24.34.112.235, 5003} & \texttt{192.168.1.2, 3345}\\ 
      \texttt{24.34.112.235, 5004} & \texttt{192.168.1.2, 3346}\\ 
      \texttt{24.34.112.235, 5005} & \texttt{192.168.1.3, 3345}\\ 
      \texttt{24.34.112.235, 5006} & \texttt{192.168.1.3, 3346}\\ 
      \end{tabular}
    \end{center}

    \item a) \begin{tabular}{l|l|l|l|l|l|l|l}
      0 & x & 6,x & 7,x & 4,x & $\infty$ & $\infty$ & $\infty$ \\
      1 & xw & 6,x & 1,w & & $\infty$ & 5,w & $\infty$ \\
      2 & xwv & 6,x & & & 6,v & 5,w & $\infty$ \\
      3 & xwvu & 6,x & & & 3,u & & $\infty$ \\
      4 & xwvuz & 2,z & & & & & 8,z \\
      5 & xwvuzy & & & & & & 4,y \\
      6 & xwvuzyt & & & & & & \\
    \end{tabular}\\
    b) The tree consists of two paths: x--w--v, x--w--u--z--y--t from a). Thus, from x, every entry is to w.\\
    c) Internet may propagate wrong link during its update.
    \item a) Since 3c learns about prefix x from the router 4c, the routing protocol corresponding to it is eBGP. \\
    b) Since 1d learns about prefix x from a router in AS1, the routing protocol corresponding to it is RIP. \\
    c) The signal will be transmitted from AS4---AS3---1c---1a---1d, so l will be equal to l1. \\
    d) Since 1c---1a---1d yields more intra-network cost than 1b---1d, via hot potato routing, l will be set to l2.
    \item SDN, software defined network, is the network whose controller computes forwarding tables and interacts with routers is implemented in a software. Other key characteristics of SDN include flow-based forwarding, separation of data plane and control plane, and programmability. 
    \item One of the attributes of the knowledge plane (KP) is edge involvement, which means KP reaches more broadly than the end-to-end principle. Besides that, KP can extend itself to the whole global world ideally. Also, multiple KP's can make a composition.
    \item a) 6 nodes are lined up on a straight line with the gap of 35 m between two adjacent nodes. \\
    b) Nodes 1, 2, 4, 5 had changes in their routing tables.

    \begin{tabular}{c|c}
    Node & One-hop connections \\ \hline
    Node 0 & [Node 1] \\ 
    Node 1 & [Node 0, Node 2, Node 4]\\ 
    Node 2 & [Node 1, Node 3, Node 4]\\ 
    Node 3 & [Node 2, Node 4]\\ 
    Node 4 & [Node 1, Node 2, Node 3]\\ 
    Node 5 & []\\ 
    \end{tabular}

    c) Node 0 and node 6.
  \end{enumerate}
}

\solution{
  \begin{enumerate}[label={\arabic*)}]
    \item a) No, because Host E can know that Host F is in the same LAN as host E by checking the subnet mask. Thus,
    \begin{itemize}
      \item source IP: IP of Host E
      \item dest'n IP: IP of Host F
      \item source MAC address: MAC address of Host E
      \item dest'n MAC address: MAC address of Host F
    \end{itemize}
    b) No, because Host E does not use MAC address of Host B and it sends an IP datagram to the router R1. In this case,
    \begin{itemize}
      \item source IP: IP of Host E
      \item dest'n IP: IP of Host B
      \item source MAC address: MAC address of Host E
      \item dest'n MAC address: MAC address of R1
    \end{itemize}
    c) Since the message is a broadcast message, S1 will broadcast the ARP request message through both links to Subnet 1 and Subnet 2. Therefore, R1 will receive the ARP request message. However, it will not be forwarded to Subnet 3.

    Host B will not send an ARP query message to ask for A's MAC address, because Host B already knows the MAC address of Host A from the query message that Host A had sent.

    S1 will add Host A in the forwarding table, and it will drop the frame because Host B and Host A are in the same subnetwork.
    
    \item 
    \item From EE side, the datagram is sent to the router. The router checks the VLAN ID and it transmits to the CS side and the CS host receives the datagram.
  \end{enumerate}
}

\solution{ 
  \begin{enumerate}[label={\arabic*)}]
    \item CSMA/CA is introduced instead.
    \item The time required to send a frame without data is $T_0=\,$256 bits / 11 Mbps = 23.273 microseconds, and the time required to send a frame with data is $T_1=\,$(8000 + 256) bits / 11 Mbps = 750.545 microseconds. Therefore, the total time required is $DIFS + T_0 + SIFS + T_0 + SIFS + T_1 + SIFS + T_0 = DIFS + 3\times SIFS + 820.364 \text{\,\mu s}.$ 
    \item a. Attacks that steals the message during its transition can be blocked, such as man-in-the-middle. \\ b. Public key is open to all, but shared key is shared with two parties sending and receiving messages.
  \end{enumerate}
}

\solution{

}

\textbf{Discussion and Suggestion} Thank you for the entire class.





























\end{document}



%1. In TCP, when congestion occurs, every end node needs to send retransmission packets to make sure that its transport data are delivered correctly. => F
%
%2. In TCP, all the congestion mechanisms require packet loss detection. =>T
%
%3. Slow start mechanism increases the number of transmitted packets linearly until the first loss event. =>F
%
%4. Every host on the network is responsible for building the network map. =>T
%
%5. _____(A)____ refers to all functions and processes that determine which path to use to send the packet or frame. _____(B)_____ refers to all the %functions and processes that forward packets/frames from one interface to another. =>(A) Control plane, (B) Data plane 
%
%
%
%
%
%1. DHCP can return more than just an allocated IP address on the subnet. => T
%
%2.A router may have multiple IP addresses when it has multiple interfaces. => T
%
%3. Forwarding is about moving a packet from a router's input port to the appropriate output port. => T
%
%4. When IPv6 tunnels through IPv4 routers, IPv6 treats the IPv4 tunnels as link-layer protocols. => T
%
%5. Suppose there are three routers between a source host and a destination host. Ignoring fragmentation, an IP datagram sent from the source host to the %destination host will travel over 9 interfaces and 3 forwarding tables will be indexed to move the datagram from the source to the destination.  => F
%
%
%1. Distance vector routing nodes determines the shortest path to any destination. => T
%
%2. In the link-state (LS) approach nodes exchange topology information only with their neighbors. =>F
%
%3. BGP is the de facto inter-domain routing protocol. =>T
%
%4. OSPF supports a hierarchical architecture for large domains. =>T
%
%5. The SDN architecture couples the control of data plane forwarding from the decision processes related to distributed applications and policy %applications. => F
%
%
%1. Link layer mainly deals with framing, error control, flow control, and channel coding. => F
%
%2. If all the links in the Internet were to provide reliable delivery service, the TCP reliable delivery service is redundant. =>F
%
%3. In CSMA/CD, after the fifth collision, what is the probability that a node chooses K=4? =>  1/32
%
%4.  An ARP query is sent within a frame with a specific destination MAC address and ARP response is sent within a broadcast frame. => F
%
%5.Internet Control Message Protocol(ICMP) has designed to compensate for error-reporting, error-correction, and host and management queries. => T
%
%6. A mobile ad hoc network (MANET) is a continuously self-configuring, self-organizing, infrastructure network of mobile devices connected without wires. => %F
%
%
%
%
%
%
%1. Before an 802.11 station transmits a data frame, it must first send an RTS frame and receive a corresponding CTS frame. =>F
%
%2. In  3G  architecture,  there  are  separate  network  components  and  paths  for  voice  and data. => T
%
%3. Ethernet and 802.11 use the same frame structure. => F
%
%4. A  permanent  address  for  a  mobile  node  is  its  IP  address  when  it  is  at  its  home network.  A  care-of-address  is  the  one  its  gets  %when  it  is  visiting  a  foreign  network. => T
%
%5. 5G mmWave system is a network architecture that enables the multiplexing of virtualized and independent logical networks on the same physical network %infrastructure. => F (mmWave system => network slicing)