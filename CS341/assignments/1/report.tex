\documentclass{homework}
% \usepackage{lua-visual-debug}
\usepackage[a4paper, total={6in, 8in}]{geometry}

\title{Reading Assignment \#1}
\subject{CS341 Introduction to Computer Networks}
\studentid{20170058}
\name{Keonwoo Kim}
\date{\today}

\setstretch{1.0}

\begin{document}
\maketitle



A network is made of entities sharing data each other. These entities are not necessarily computing machines, where the terminology 'computer networks' points ones consisting of computers. Some real life examples give us an inspiration to a concept in computer networks. For instance, we can think of the network in a restaurant. A client, who is a customer of the restaurant, will order a menu to the server, who is a waiter. Then the waiter got the desired menu anyway and they deliver the menu to the client. Besides this, there are lots of networks in our real world, such as a family network, a peer network where the family network gets connected with to form a bigger society, a contact network to get a job, etc.

However, in particular, we are interested in computer networks, which are networks where the entities forming the networks are computers. Your computer, as one of the computers in the network, is called the local computer.

To use the computer networks in a manageable way, a network plan is needed. It has the informations about how to connect to which devices and share what types of data with them. This involves a management of informations such as considering importances of informations, or giving some authorization processes to the data sharing requests to concern about the security. Usually, as the hierarchy of information shows, the more detailed information is, the more limited it is ought to be.

By enabling networking, we can make our computers more powerful and more efficient because it is not necessary for every machine to have a copy of every information or peripherals as any machine can share and borrow data or devices via the computer network. Before the network services being popular, people need to carry storage devices with their own, where this process is called a sneakernet, so that it was difficult and time-consuming to share the data with the computers far from the local one. However, the advent of networking techniques significantly reduced the hurdles of time and distance. Computer networks also allowed people to collaborate each other with their computers.

To manage the networks and softwares with lower total cost of ownership--the total amount of resources that companies spend on an issue during its lifetime, people started to standardize the components and make computers interoperate each other. The networking techniques made the cost to make the same system much lower and made the overall system fairly robust and easier to maintain through installing software remotely, equalizing the equipments used as network devices so that it became easier to maintain the network, and backing up necessary files over the network.

\end{document}